% https://groups.google.com/forum/#!topic/comp.text.tex/s6z9Ult_zds
\makeatletter\let\ifGm@compatii\relax\makeatother

\ifx\notes\undefined
    \documentclass{beamer}
 \else
   \documentclass[handout]{beamer}
    \mode<handout>{
        \usepackage{pgfpages}
    \pgfpagesuselayout{4 on 1}[a4paper ,border shrink=2mm, landscape]
	\pgfpageslogicalpageoptions{1}{border code=\pgfsetlinewidth{1mm}\pgfusepath{stroke}}
	\pgfpageslogicalpageoptions{2}{border code=\pgfsetlinewidth{1mm}\pgfusepath{stroke}}
	\pgfpageslogicalpageoptions{3}{border code=\pgfsetlinewidth{1mm}\pgfusepath{stroke}}
	\pgfpageslogicalpageoptions{4}{border code=\pgfsetlinewidth{1mm}\pgfusepath{stroke}}
%    \setbeamercolor{background canvas}{bg=black!3}
    }
\fi
%http://www.cpt.univ-mrs.fr/~masson/latex/Beamer-appearance-cheat-sheet.pdf
\definecolor{Dandelion}         {RGB}{253, 200, 47}
\setbeamertemplate{frametitle}
{\vspace{1mm}
\begin{centering}
\insertframetitle
\end{centering}
{\color{Dandelion} \rule{\textwidth}{1mm}}
}
\setbeamertemplate{itemize item}{$\bullet$}
\setbeamertemplate{navigation symbols}{}
%\setbeamercolor{title}{fg=Fern}
%\setbeamercolor{frametitle}{fg=Fern}
%\setbeamercolor{normal text}{fg=Charcoal}
%\setbeamercolor{block title}{fg=black,bg=Fern!25!white}
%\setbeamercolor{block body}{fg=black,bg=Fern!25!white}
%\setbeamercolor{alerted text}{fg=AlertColor}
\setbeamercolor{itemize item}{fg=black}

\usepackage[latin1]{inputenc}
\usepackage{epic,eepic,epsf}
%\usepackage{latexsym}
\usepackage{verbatim}
\usepackage{listings}
\usepackage{alltt}
\usepackage{minted}

%% all code examples use "code" environment
\newenvironment{code}{\begin{minted}{c}}{\end{minted}}

%\newcommand{\UnderscoreCommands}{\do\verbatiminput%
%\do\citeNP \do\citeA \do\citeANP \do\citeN \do\shortcite%
%\do\shortciteNP \do\shortciteA \do\shortciteANP \do\shortciteN%
%\do\citeyear \do\citeyearNP%
%}
%\usepackage[strings]{underscore}
%\usetheme{Pittsburgh}

%\makeatletter
%\setbeamercolor{background canvas}{bg=white}
%\setbeamercolor{background}{bg=white}
%\makeatother

\title[COMP1511 Introduction to Programming]{COMP1511}
\author{Andrew Taylor/John Shepherd}
\institute{CSE, UNSW}
\date{\today}

\newcommand{\red}[1]{\textcolor{red}{#1}}
\newcommand{\blue}[1]{\textcolor{blue}{#1}}
\newcommand<>{\RedNote}[2]{%
  \begin{alertblock}#3{#1}
    #2
  \end{alertblock}}
\newcommand<>{\Note}[2]{%
  \begin{exampleblock}#3{#1}
    #2
  \end{exampleblock}}
\newcommand{\NotesFrame}{\frame<handout>[plain]{\frametitle{My Notes}}}
\newcommand{\ttblue}[1]{\texttt{\blue{#1}}}

\usepackage{epstopdf}


% prefer pdf over png if both versions exist
\DeclareGraphicsExtensions{%
    .pdf,.PDF,%
    .png,.PNG,%
    .jpg,.mps,.jpeg,.jbig2,.jb2,.JPG,.JPEG,.JBIG2,.JB2}

\setbeamercolor{block body}{bg=gray!7}

\newenvironment{C}
 {\VerbatimEnvironment
  \setbeamertemplate{blocks}[rounded][shadow=true]
  \begin{block}{}
  \begin{minted}{c}}
 {\end{minted}
  \end{block}}
  
\newenvironment{sh}
 {\VerbatimEnvironment
  \setbeamertemplate{blocks}[rounded][shadow=true]
  \begin{block}{}
  \begin{minted}{sh}}
 {\end{minted}
  \end{block}}
  
\newenvironment{perl}
 {\VerbatimEnvironment
  \setbeamertemplate{blocks}[rounded][shadow=true]
  \begin{block}{}
  \begin{minted}{perl}}
 {\end{minted}
  \end{block}}
  
\newenvironment{python}
 {\VerbatimEnvironment
  \setbeamertemplate{blocks}[rounded][shadow=true]
  \begin{block}{}
  \begin{minted}{python}}
 {\end{minted}
  \end{block}}

\begin{document}
\section{Perl - Introduction}

\begin{frame}
\frametitle{Perl}
Perl = {\bf{P}}ractical {\bf{E}}xtraction and {\bf{R}}eport {\bf{L}}anguage

Developed by Larry Wall (late 80's) as a replacement for \textbf{\tt{awk}}.

Has grown to become a replacement for \textbf{\tt{awk}}, \textbf{\tt{sed}},
\textbf{\tt{grep}}, other filters, shell scripts, C programs, ...
(i.e. "kitchen sink").

An extremely useful tool to know because it:
\begin{itemize}
\item  runs on Unix variants (Linux/Android/OSX/IOS/..), Windows/DOS variants, Plan 9, OS2, OS390, VMS..
\item  very widely used for complex scripting tasks 
\item  has standard libraries for many applications (Web/CGI, DB, ...)
\end{itemize}
Perl has been influential: PHP, Python, Ruby, ... even Java {\small (interpreted)}
\end{frame}

\begin{frame}
\frametitle{Perl}
Some of the language design principles for Perl:
\begin{itemize}
\item  make it easy/concise to express common idioms 
\item  provide many different ways to express the same thing
\item  use defaults where every possible
\item  exploit powerful concise syntax \& accept ambiguity/obscurity in some cases
\item  create a large language that users will learn subsets of
\end{itemize}
{\small Many of these conflict with design principles of languages for teaching.}
\end{frame}

\begin{frame}
\frametitle{Perl}
So what is the end product like?
\begin{itemize}
\item  a language which makes it easy to build useful systems
\item  readability can sometimes be a problem ~ {\small (language is too rich?)}
\item  interpreted ~ efficient? ~ {\small (although still remarkably efficient)}
\end{itemize}
Summary: it's easy to write concise, powerful, obscure programs in Perl
\end{frame}

\begin{frame}[shrink]
\frametitle{Reference Material}

\begin{itemize}
\item  {\em{Wall, Christiansen \& Orwant }},
	Programming Perl (3ed),
	\\{\small O'Reilly, 2000.
		~ (Original \& best Perl reference manual)}
\item  {\em{Schwartz, Phoenix \& Foy}},
	Learning Perl (5ed),
	\\{\small O'Reilly, 2008.
		~ (gentle \&  careful introduction to Perl)}
\item  {\em{Christiansen \& Torkington}},
	Perl Cookbook (2ed),
	\\{\small O'Reilly, 2003.
		~ (Lots and lots of interesting Perl examples)}
\item  {\em{Schwartz \& Phoenix}},
	Learning Perl Objects, References, and Modules (2ed),
	\\{\small O'Reilly, 2003.
		~ (gentle \&  careful introduction to parts of Perl mostly  not covered in this course)}
\item  {\em{Schwartz, Phoenix \& Foy}},
	Intermediate Perl  (2ed),
	\\{\small O'Reilly, 2008.
		~ (good book to read after 2041 - starts where this course finishes)}
\item  {\em{Sebesta}},
	A Little Book on Perl,
	\\{\small Prentice Hall, 1999.
		~ (Modern, concise introduction to Perl)}
\item  {\em{Orwant, Hietaniemi, MacDonald}},
	Mastering Algorithms with Perl,
	\\{\small O'Reilly, 1999.
		~ (Algorithms and data structures via Perl)}
\end{itemize}
\end{frame}

\begin{frame}[fragile]
\frametitle{Running Perl}

Perl programs can be invoked in several ways ...

\begin{itemize}
\item  giving the filename of the Perl program as a command line argument:

\begin{sh}
    perl PerlCodeFile.pl
\end{sh}

\item  giving the Perl program itself as a command line argument:

\begin{sh}
    perl -e 'print "Hello, world\n";'
\end{sh}

\item  using the \textbf{\tt{\#!}} notation and making the program file executable:

\begin{sh}
    chmod 755 PerlCodeFile
    ./PerlCodeFile
\end{sh}

\end{itemize}
\end{frame}

\begin{frame}[fragile]
\frametitle{Running Perl}
Advisable to {\it{always}} use \textbf{\tt{-w}} option.

Causes Perl to print warnings about common errors.
\begin{sh}
    perl -w PerlCodeFile.pl
    perl -w -e 'PerlCode'
\end{sh}

Can use options with \textbf{\tt{\#!}}
\begin{perl}
    #!/usr/bin/perl -w
    
    PerlCode
\end{perl}

you can also get warnings via a pragma:

\begin{perl}
    use warnings;
\end{perl}

To catch other possible problems.
Some programmers always use strict, others find it too annoying.

\begin{perl}
    use strict;
\end{perl}

\end{frame}

\begin{frame}
\frametitle{Syntax Conventions}
Perl uses non-alphabetic characters to introduce various kinds of
program entities
{\small (i.e. set a context in which to interpret identifiers)}.

\begin{center}
\begin{tabular}{llll}

  {\bf{Char}} & 
  {\bf{Kind}} & 
  {\bf{Example}} & 
  {\bf{Description}}
\\

  {\bf{\textbf{\tt{\#}}}} & 
  Comment & 
  \textbf{\tt{\# comment}} & 
  rest of line is a comment
\\

  {\bf{\textbf{\tt{\$}}}} & 
  Scalar & 
  \textbf{\tt{\$count}} & 
  variable containing simple value
\\

  {\bf{\textbf{\tt{@}}}} & 
  Array & 
  \textbf{\tt{@counts}} & 
  list of values, indexed by integers
\\

  {\bf{\textbf{\tt{\%}}}} & 
  Hash & 
  \textbf{\tt{\%marks}} & 
  set of values, indexed by $strings$
\\

  {\bf{\textbf{\tt{\&}}}} & 
  Subroutine & 
  \textbf{\tt{\&doIt}} & 
  callable Perl code
  (\& optional)
\\
\end{tabular}
\end{center}
~\\\vspace{1ex}
\end{frame}

\begin{frame}[fragile]
\frametitle{Syntax Conventions}
Any unadorned identifiers are either
\begin{itemize}
\item  names of built in (or other) functions ~ (e.g. \textbf{\tt{chomp}}, \textbf{\tt{split}})
\item  control-structures ~ (e.g. \textbf{\tt{if}}, \textbf{\tt{for}}, \textbf{\tt{foreach}})
\item  literal strings (like the shell!)
\end{itemize}
The latter can be confusing to C/Java/PHP programmers e.g.
\begin{perl}
    $x = abc;   is the same as   $x = "abc";
\end{perl}

\end{frame}

\begin{frame}
\frametitle{Variables}
Perl provides these basic kinds of variable:

\begin{itemize}
\item  {\em{scalars}} ... a single atomic value (number or string)
\item  {\em{arrays}} ... a list of values, indexed by number
\item  {\em{hashes}} ... a group of values, indexed by string
\end{itemize}

Variables do not need to be declared or initialised.

If not initialised, a scalar is the empty string {\small (0 in a numeric context)}.

{\em{Beware:}} spelling mistakes in variable names, e.g:

\begin{center}
\textbf{\tt{print "abc=\$acb{\textbackslash}n";}}  ~~~ rather than ~~~ \textbf{\tt{print "abc=\$abc{\textbackslash}n";}}
\end{center}

Use warnings (-w) and easy to spell variable names.

\end{frame}

\begin{frame}
\frametitle{Variables}
Many scalar operations have a "default source/target".

If you don't specify an argument,  variable \textbf{\tt{\$\_}} is assumed

This makes it
\begin{itemize}
\item  often very convenient to write brief programs
	~ {\small (minimal syntax)}
\item  sometimes confusing to new users
	~ {\small ("Where's the argument??")}
\end{itemize}

\textbf{\tt{\$\_}} performs a similar role to "it" in English text.

E.g. ``The dog ran away. It ate a bone. It had lots of fun.''

\end{frame}

\begin{frame}
\frametitle{Arithmetic \& Logical Operators}

Perl arithmetic and logical operators are similar to C.

Numeric: \textbf{\tt{==  !=  <  <=  >  >=  <=>}}

String: \textbf{\tt{eq  ne  lt  le  gt  ge  cmp}}

Most C operators are present and have similar meanings, e.g: \\
\textbf{\tt{+ - * / \% ++ -- += }}

Perl string concatenation operator: \textbf{\tt{.}} \\
equivalent to using C's \tt{malloc} + \tt{strcat}

C \tt{strcmp} equivalent to Perl cmp

\end{frame}

\begin{frame}[fragile,shrink]
\frametitle{Scalars}
Examples:
\begin{perl}
$x = '123';    # $x assigned string "123"
$y = "123 ";   # $y assigned string "123 "
$z = 123;      # $z assigned integer 123
$i = $x + 1;   # $x value converted to integer
$j = $y + $z;  # $y value converted to integer
$a = $x == $y; # numeric compare $x,$y  (true)
$b = $x eq $y; # string compare  $x,$y   (false)
$c = $x.$y;    # concat $x,$y (explicit)
$c = "$x$y";   # concat  $x,$y (interpolation)
\end{perl}

Note: \textbf{\tt{\$c = \$x \$y}} is invalid
{\small (Perl has no empty infix operator)}

{\small (unlike predecessor languages such as awk, where \textbf{\tt{\$x \$y}} meant string concatenation)}
\end{frame}

\begin{frame}
\frametitle{ Perl Truth Values}

False: '' and '0'

True: everything else.

Be careful, subtle consequences:

False: 0.0, 0x0

True: '0.0' and "0{\textbackslash}n"


\end{frame}

\begin{frame}[fragile]
\frametitle{Scalars}
A very common pattern for modifying scalars is:
\begin{perl}
    $var = $var op expression
\end{perl}

{\em{Compound assignments}} for the most common operators allow you to write
\begin{perl}
    $var op= expression
\end{perl}

Examples:
\begin{perl}
    $x += 1;     # increment the value of $x
    $y *= 2;     # double the value of $y
    $a .= "abc"  # append "abc" to $a
\end{perl}

\end{frame}

\begin{frame}
\frametitle{Logical Operators}
Perl has two sets of logical operators, one like C, the other like "English".

The second set has very low precedence, so can be used between statements.

\begin{center}
\begin{tabular}{lllll}

  {\bf{Operation}} & 
  ~ & 
  {\bf{Example}} & 
  ~ & 
  {\bf{Meaning}}
\\

  And & 
  ~ & 
  \textbf{\tt{x \&\& y}} & 
  ~ & 
  false if \textbf{\tt{x}} is false, otherwise \textbf{\tt{y}}
\\

  Or & 
  ~ & 
  \textbf{\tt{x || y}} & 
  ~ & 
  true if \textbf{\tt{x}} is true, otherwise \textbf{\tt{y}}
\\

  Not & 
  ~ & 
  \textbf{\tt{! x}} & 
  ~ & 
  true if \textbf{\tt{x}} is not true, false otherwise
\\

  And & 
  ~ & 
  \textbf{\tt{x and y}} & 
  ~ & 
  false if \textbf{\tt{x}} is false, otherwise \textbf{\tt{y}}
\\

  Or & 
  ~ & 
  \textbf{\tt{x or y}} & 
  ~ & 
  true if \textbf{\tt{x}} is true, otherwise \textbf{\tt{y}}
\\

  Not & 
  ~ & 
  \textbf{\tt{not}} x & 
  ~ & 
  true if \textbf{\tt{x}} is not true, false otherwise
\\
\end{tabular}
\end{center}
\end{frame}

\begin{frame}[fragile]
\frametitle{Logical Operators}

The lower precedence of or/and enables
common Perl idions. 

\begin{perl}
if (!open FILE, '<', "a.txt") {
    die "Can't open a.txt: $!";
}
\end{perl}


is often replaced by Perl idiom

\begin{perl}
open FILE, '<', "a" or die "Can't open a: $!";
\end{perl}

Note this doesn't work:

\begin{perl}
open FILE, '<', "a" || die "Can't open a: $!";
\end{perl}

because its equivalent to:

\begin{perl}
open FILE, '<', ("a" || die "Can't open a: $!");
\end{perl}


\end{frame}

\begin{frame}[fragile,shrink]
\frametitle{Stream Handles}

Input \& output are accessed via {\em{handles}} - similar to \textbf{\tt{FILE *}} in C.


\begin{perl}
$line = <IN>;  # read next line from stream IN
\end{perl}


Output file handles can be used as the first argument to \textbf{\tt{print}}:

\begin{perl}
print OUT "Andrew\n";  # write line to stream OUT
\end{perl}

{\small Note: no comma after the handle}

Predefined handles for stdin, stdout, stderr

\begin{perl}
# STDOUT is default for print so can be omitted
print STDOUT "Enter your a number: ";
$number = <STDIN>;
if (number < 0) {
    print STDERR "bad number\n";
}
\end{perl}

\end{frame}

%%%\begin{frame}[fragile,shrink]
%%%\frametitle{Input/Output}
%%%
%%%Example (a simple \textbf{\tt{cat}}):
%%%
%%%\begin{perl}
%%%#!/usr/bin/perl
%%%# Copy stdin to stdout
%%%
%%%while ($line = <STDIN>) {
%%%    print $line;
%%%}
%%%\end{perl}
%%%
%%%However, this can be simplified to:
%%%\begin{perl}
%%%while (<STDIN>) { print; }
%%%
%%%# or even
%%%
%%%print <STDIN>;
%%%\end{perl}
%%%
%%%Defaults:
%%%\begin{itemize}
%%%\item  the default destination variable for input is \textbf{\tt{\$\_}}
%%%\item  the default argument for \textbf{\tt{print}} is also \textbf{\tt{\$\_}}
%%%\end{itemize}
%%%\end{frame}

\begin{frame}[fragile,shrink]
\frametitle{Opening Files}
Handles can be explicitly attached to files via the \textbf{\tt{open}} command:

\begin{perl}
open DATA, '<', 'data';   # read from file data
open RES, '>',  results'; # write to file results
open XTRA, '>>', stuff";  # append to file stuff"
\end{perl}

Handles can even be attached to pipelines to read/write to Unix commands:
\begin{perl}
open DATE, "date|";  # read output of date command
open FEED, "|more"); # send output through "more"
\end{perl}

Opening a file may fail - always check:

\begin{perl}
open DATA, '<', 'data' or  die "Can't open data: $!";
\end{perl}
\end{frame}

\begin{frame}[fragile,shrink]
\frametitle{Reading and Writing a File: Example}

\begin{perl}
open OUT, '>', 'a.txt' or die "Can't open a.txt: $!";
print OUT "42\n";
close OUT;
open IN, '<', 'a.txt' or die "Can't open a.txt: $!";
$answer = <IN>;
close IN;
print "$answer\n"; # prints 42
\end{perl}
\end{frame}

\begin{frame}[fragile,shrink]
\frametitle{Anonymous File Handles}

If you supply a uninitialized variable Perl will 
store  an anonymous file handle in it:

\begin{perl}
open my $output, '>', 'answer' or die "Can't open ...";
print $output "42\n";
close $output;
open my $input, '<', 'answer' or die "Can't open ...";
$answer = <$input>;
close $input;
print "$answer\n"; # prints 42
\end{perl}

Use this approach for larger programs to avoid
collision between file handle names.

\end{frame}

\begin{frame}[fragile,shrink]
\frametitle{Close}

Handles can be explitly closed with ~\textbf{\tt{close(}}$HandleName$\textbf{\tt{)}}

\begin{itemize}
\item
All handles closed on exit.
\item
Handle also closed if open done with same name \\
good for lazy coders.
\item
Data on output streams may be not written (buffered) \\
until close - hence close  ASAP.
\end{itemize}

\end{frame}

\begin{frame}[fragile,shrink]
\frametitle{\textbf{\tt{{\textless}>}} give Unix Filter behavior}
Calling \textbf{\tt{{\textless}>}} without a file handle gets unix-filter behaviour.
\begin{itemize}
\item  treats all command-line arguments as file names
\item  opens and reads from each of them in turn
\item  no command line arguments, then ~ \textbf{\tt{{\textless}> == {\textless}STDIN>}}
\end{itemize}

So this is \tt{cat} in Perl:

\begin{perl}
%%%#!/usr/bin/perl
%%%# Copy stdin to stdout
%%%
%%%while ($line = <>) {
%%%    print $line;
%%%}
\end{perl}

Displays the contents of the files \textbf{\tt{a}}, \textbf{\tt{b}}, \textbf{\tt{c}} on stdout.
\end{frame}

\begin{frame}[fragile]
\frametitle{Control Structures}
{\em{All}} single Perl statements must be terminated by a semicolon, e.g.
\begin{perl}
    $x = 1;
    print "Hello";
\end{perl}

{\em{All}} statements with control structures must be enclosed in braces, e.g.
\begin{perl}
    if ($x > 9999) {
        print "x is big\n";
    }
\end{perl}

You don't need a semicolon after a statement group in \textbf{\tt{\{...\}}}.

Statement blocks can also be used like anonymous functions.
\end{frame}

\begin{frame}[fragile]
\frametitle{Function Calls}
All Perl function calls ...
\begin{itemize}
\item  are call by value ~ {\small (like C)} (except  scalars aliased to @\_)
\item  are expressions ~ {\small (although often ignore return value)}
\end{itemize}

Notation(s) for Perl function calls:
\begin{perl}
    &func(arg{1}, arg{2}, ... arg{n})
    func(arg{1}, arg{2}, ... arg{n})
    func arg{1}, arg{2}, ... arg{n}
\end{perl}

\end{frame}

\begin{frame}[fragile,shrink]
\frametitle{Control Structures}
Selection is handled by ~ \textbf{\tt{if}} ... \textbf{\tt{elsif}} ... \textbf{\tt{else}}
\begin{perl}
    if ( boolExpr{1} ) { statements{1} }
    elsif ( boolExpr{2} ) { statements{2} }
    ...
    else { statements{n} }
    
    statement if ( expression );
\end{perl}

\end{frame}

\begin{frame}[fragile,shrink]
\frametitle{Control Structures}
Iteration is handled by ~ \textbf{\tt{while}}, \textbf{\tt{until}}, \textbf{\tt{for}}, \textbf{\tt{foreach}}
\begin{perl}
    while ( boolExpr ) {
        statements
    }
    
    until ( boolExpr ) {
        statements
    }
    
    for ( init ; boolExpr ; step ) {
        statements
    }                    
    
    foreach var (list) {
        statements
    }
\end{perl}

\end{frame}

\begin{frame}[fragile,shrink]
\frametitle{Control Structures}
Example (compute $pow = k^{n}$):
\begin{perl}
    # Method 1 ... while
    $pow = $i = 1;
    while ($i <= $n) {
        $pow = $pow * $k;
        $i++;
    }
\end{perl}

\begin{perl}
    # Method 2 ... for
    $pow = 1;
    for ($i = 1; $i <= $n; $i++) {
        $pow *= $k;
    }
\end{perl}

\begin{perl}
    # Method 3 ... foreach
    $pow = 1;
    foreach $i (1..$n) {
    	$pow *= $k;
    }
\end{perl}

\begin{perl}
    # Method 4 ... builtin operator
    $pow = $k ** $n;
\end{perl}

\end{frame}

\begin{frame}[fragile,shrink]
\frametitle{Control Structures}
Example (compute $pow = k^{n}$):
\begin{perl}
    # Method 1 ... while
    $pow = $i = 1;
    while ($i <= $n) {
        $pow = $pow * $k;
        $i++;
    }
\end{perl}

\begin{perl}
    # Method 2 ... for
    $pow = 1;
    for ($i = 1; $i <= $n; $i++) {
        $pow *= $k;
    }
\end{perl}

\begin{perl}
    # Method 3 ... foreach
    $pow = 1;
    foreach $i (1..$n) { $pow *= $k; }
\end{perl}

\begin{perl}
    # Method 4 ... foreach $_
    $pow = 1;
    foreach (1..$n) { $pow *= $k; }
\end{perl}

\begin{perl}
    # Method 5 ... builtin operator
    $pow = $k ** $n;
\end{perl}

\end{frame}

%%%\begin{frame}[fragile,shrink]
%%%\frametitle{Control Structures}
%%%Example of fine-grained loop control:
%%%\begin{perl}
%%%OUTER:
%%%while (i < 1000)
%%%{
%%%    INNER:
%%%    for ($i = 0; $i < 99; $i++)
%%%    {
%%%        last OUTER if $i > 90;   # terminates both loops
%%%        $i += 3;
%%%        next if $i > 80;         # next iteration of INNER
%%%        if ($i > 70) { next; }   # next iteration of INNER
%%%        $i += 4;
%%%        redo if $i < 70;         # next iteration of INNER
%%%        next OUTER if $i == 42;  # next iteration of OUTER
%%%     }
%%%}
%%%\end{perl}
%%%
%%%\end{frame}

\begin{frame}[fragile,shrink]
\frametitle{Terminating}
Normal termination, call: \textbf{\tt{exit 0}} 

The \textbf{\tt{die}} function is used for abnormal termination:
\begin{itemize}
\item  accepts a list of arguments
\item  concatenates them all into a single string
\item  appends file name and line number
\item  prints this string
\item  and then terminates the Perl interpreter
\end{itemize}
Example:
\begin{perl}
    if (! -r "myFile") {
        die "Can't read myFile: $!\n";
    }
    # or
    die "Can't read myFile: $!\n" if ! -r "myFile";
    # or
    -r "myFile" or die "Can't read myFile: $!\n"
\end{perl}

\end{frame}

\begin{frame}[fragile,shrink]
\frametitle{Perl and External Commands}
Perl is shell-like in the ease of invoking other commands/programs.

Several ways of interacting with external commands/programs:

\begin{center}
\begin{tabular}{lll}

\begin{minipage}{3cm} \textbf{\tt{{\em{`}}}}$cmd$\textbf{\tt{{\em{`}};}}  ~\end{minipage}
 & \begin{minipage}{18cm} capture entire output of $cmd$ as single string ~\end{minipage}
\\[1ex]

\begin{minipage}{3cm} \textbf{\tt{system "}}$cmd$\textbf{\tt{"}}  ~\end{minipage}
 & \begin{minipage}{18cm} execute $cmd$ and capture its exit status only ~\end{minipage}
\\[1ex]

\begin{minipage}{3cm} \textbf{\tt{open F,"}}$cmd$\textbf{\tt{|"}}  ~\end{minipage}
 & \begin{minipage}{18cm} collect $cmd$ output by reading from a stream ~\end{minipage}
\\[1ex]
\end{tabular}
\end{center}

\end{frame}

\begin{frame}[fragile,shrink]
\frametitle{Perl and External Commands}
External command examples:
\begin{perl}
$files = `ls $d`;  # output captured

$exit_status = system "ls $d"; # output to stdout

open my $files, '-|', "ls $d";  # output to stream
while (<$files>) {
    chomp;
    @fields = split;    # split words in $_ to @_
    print "Next file is $fields[$#fields]\n";
}
\end{perl}

\end{frame}

\begin{frame}[fragile,shrink]
\frametitle{File Test Operators}
Perl provides an extensive set of operators to query file information:

\begin{center}
\begin{tabular}{lll}

  \begin{minipage}{3cm}\textbf{\tt{-r, -w, -x}} ~\end{minipage}
   & \begin{minipage}{18cm}file is readable, writeable, executable~\end{minipage}
\\[1ex]

  \begin{minipage}{3cm}\textbf{\tt{-e, -z, -s}} ~\end{minipage}
   & \begin{minipage}{18cm}file exists, has zero size, has non-zero size~\end{minipage}
\\[1ex]

  \begin{minipage}{3cm}\textbf{\tt{-f, -d, -l}} ~\end{minipage}
   & \begin{minipage}{18cm}file is a plain file, directory, sym link~\end{minipage}
\\[1ex]
\end{tabular}
\end{center}

Cf. the Unix \textbf{\tt{test}} command.

Used in checking I/O operations, e.g.
\begin{perl}
    -r "dataFile" && open my $data, '<', "dataFile";
\end{perl}

\end{frame}

\begin{frame}[fragile,shrink]
\frametitle{Special Variables}
Perl defines numerous special variables to hold information about its
execution environment.

These variables typically have names consisting of a single punctuation
character e.g. ~ \textbf{\tt{\$!  \$@  \$\#  \$\$  \$\%}} ~ ...
	~{\small (English names are also available)}

The \textbf{\tt{\$\_}} variable is particularly important:
\begin{itemize}
\item  acts as the default location to assign result values (e.g. \textbf{\tt{{\textless}STDIN>}})
\item  acts as the default argument to many operations (e.g. \textbf{\tt{print}})
\end{itemize}
Careful use of \textbf{\tt{\$\_}} can make programs concise, uncluttered. \\
Careless use of \textbf{\tt{\$\_}} can make programs cryptic.
\end{frame}

\begin{frame}[fragile,shrink]
\frametitle{Special Variables}

\begin{center}
\begin{tabular}{lll}

   \begin{minipage}{2cm}\textbf{\tt{\$\_}} ~\end{minipage}
    & \begin{minipage}{18cm}default input and pattern match~\end{minipage}
\\[1ex]

   \begin{minipage}{2cm}\textbf{\tt{@ARGV}} ~\end{minipage}
    & \begin{minipage}{18cm}list (array) of command line arguments~\end{minipage}
\\[1ex]

   \begin{minipage}{2cm}\textbf{\tt{\$0}} ~\end{minipage}
    & \begin{minipage}{18cm}name of file containing executing Perl script (cf. shell)~\end{minipage}
\\[1ex]

   \begin{minipage}{2cm}\textbf{\tt{\$}}$i$ ~\end{minipage}
    & \begin{minipage}{18cm}matching string for $i^{th}$ regexp in pattern~\end{minipage}
\\[1ex]

   \begin{minipage}{2cm}\textbf{\tt{\$!}} ~\end{minipage}
    & \begin{minipage}{18cm}last error from system call such as open~\end{minipage}
\\[1ex]

   \begin{minipage}{2cm}\textbf{\tt{\$.}} ~\end{minipage}
    & \begin{minipage}{18cm}line number for input file stream~\end{minipage}
\\[1ex]

   \begin{minipage}{2cm}\textbf{\tt{\$/}} ~\end{minipage}
    & \begin{minipage}{18cm}line separator, none if undefined~\end{minipage}
\\[1ex]

   \begin{minipage}{2cm}\textbf{\tt{\$\$}} ~\end{minipage}
    & \begin{minipage}{18cm}process number of executing Perl script (cf. shell)~\end{minipage}
\\[1ex]

   \begin{minipage}{2cm}\textbf{\tt{\%ENV}} ~\end{minipage}
    & \begin{minipage}{18cm}lookup table of environment variables~\end{minipage}
\\[1ex]
\end{tabular}
\end{center}

\end{frame}

\begin{frame}[fragile,shrink]
\frametitle{Special Variables}
Example (\textbf{\tt{echo}} in Perl):
\begin{perl}
    for ($i = 0; $i < @ARGV; $i++) {
        print "$ARGV[$i] ";
    }
    print "\n";
\end{perl}

or
\begin{perl}
    foreach $arg (@ARGV) {
        print "$arg ";
    }
    print "\n";
\end{perl}

or
\begin{perl}
    print "@ARGV\n";
\end{perl}

\end{frame}


\end{document}
